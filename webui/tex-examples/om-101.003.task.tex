Ускорение точки $A_{1}$ в любой момент промежутка [$t_{0}$,$t_{1}$]
отлично от нуля и равно удвоенному ускорению точки $A_{2}$. Можно ли 
утверждать, что $v_{A_{1}}(t_{1})=2v_{A_{2}}(t_{1})$?
