
Введенные в (1) обозначения $ F,\,S $  выражаются  следующим образом [1,2]:
$$
F=m\rho^2\Omega^{-2}A_1^{-3}A_3,\,\,S=m\rho^3\Lambda\Omega^{-3}d|d|A_1^{-4}A_3^4,\,\,d=1-A_3A_1^{-1}.\eqno(3)
$$

обозначим
$\text{ПЗ}$\,{,}
$\text{ОЗ}_{m}^{q}$\,{,}
$\text{ОЗ}_{m}^{f}$\,{,}
$\text{ОЗ}_{\left(m,\tau\right)}^{\left(q,f\right)}$\,{,}
соответственно.

и  обозначим  их
$\text{ОЭЗ}_{m}^{q}$\,{,}
$\text{ОЭЗ}_{m}^{f}$\,{,}
$\text{ОЭЗ}_{\left(m,\tau\right)}^{\left(q,f\right)}$\,{,}
соответственно.

где $\widetilde{\sigma}_n$, $n\geqslant m+1$,~--- одноточечные множества, а $~\widetilde{\sigma}_m$~--- конечное множество с числом

Здесь $g_\mu \Phi (x){=}\int\limits_0^1 g(x,t,\mu )dt$, $U_0(g_\mu
\Phi )=\int\limits_0^1 U_{0x}(g(x,t,\mu ))\Phi(t) dt$, ($U_{0x}$
означает, что $U_0$ применяется к g по переменной x), $g(x,t,\mu
)=diag(g_1(x,t,\mu ),g_2(x,t,\mu ))$,\\ $g_k(x,t,\mu )=-\varepsilon
(t,x)exp{(-1)^{k-1}\mu (x-t)}$, при $(-1)^{k-1}Re\mu \geqslant 0$,

\begin{center}{ \bf  ОБРАТНАЯ ЗАДАЧА ДЛЯ  УРАВНЕНИЯ\\ В БАНАХОВОМ ПРОСТРАНСТВЕ,\\ НЕ РАЗРЕШИМОГО ОТНОСИТЕЛЬНО ПРОИЗВОДНОЙ РИМАНА~--- ЛИУВИЛЛЯ}\\

$$
\dfrac{\pi \lambda}{\sin \pi \lambda} &\quad \text{for $0\leqslant \lambda <\dfrac{1}{2}$,}\\
$$

При $\varkappa >0$ для орбит $(a, b)$ c $a >g(b)$ особые точки $\Sigma$, не встречавшиеся при $\varkappa = 0$,~лежат в области $\{h > h_0(a, b)\}$, а остальные --- в $\{h < h_0(a, b)\}$.

где ${\cal P}^\gamma$~--- многомерный оператор Пуассона,\\

Рощупкин\,С.А. \\
Елецкий государственный университет им. И.А. Бунина, Россия \\
 {\it roshupkinsa@mail.ru}\\
 Санина \,Е.Л. \\
Воронежский государственный университет, Россия \\
{\it sanina08@mail.ru}
\end{center}

\addcontentsline{toc}{section}{Ляхов Л.Н., Рощупкин\,С.А., Санина \,Е.Л.}

будем обозначать одну или несколько арифметических операций (типа умножения, суммы, разности) между функциями $\left(^\gamma T^xf\right)(y)$\,и \,$g(y)$.

задаче на собственные значения с нелинейной зависимостью от спектрального параметра\,[1]:

3. {\it Соловьёв\,С.И.}
Собственные колебания стержня с упруго присоединённым грузом
