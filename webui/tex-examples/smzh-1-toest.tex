\documentclass{article}

\usepackage[T2A]{fontenc}
\usepackage[cp1251]{inputenc}
\usepackage[english,russian]{babel}

\newcommand{\R}{\mathbb{R}}
\renewcommand{\epsilon}{\varepsilon}

\newenvironment{nam}[2]{begdef}{enddef}
\renewenvironment{nam}[2]{begdef}{enddef}

\begin{document}


Some text with frac $\frac{1}{2}$ in it.%

Some text with frac $\frac{a}{b}+\frac{1+\alpha}{2-\beta}+\frac{1+C_0}{2}$ in it.  %comment

 % Line-long comment

%%%%%%%%%%%%%%%%%%%%%%%%%
%%% EVIL COMMENT
%%%%%%%%%%%%%%%%%%%%%%%%%

\begin{equation}\label{eq1}
E=mc^2
\end{equation}
то есть энергия есть масса (см. \cite{Einstein}).


$E=mc^2$,то есть энергия есть масса (см. \cite{Einstein}).
$E=mc^2$, то есть энергия есть масса (см. \cite{Einstein}).

$E=mc^2$,то-есть энергия есть масса (см. \cite{Einstein}).
$E=mc^2$,то~есть энергия есть масса (см. \cite{Einstein}).
$E=mc^2$, тоесть энергия есть масса (см. \cite{Einstein}).

$E=mc^2$,то-~есть энергия есть масса (см. \cite{Einstein}).


$$E=mc^2$$

Здесь и далее полагаем $c=3\cdot10^8$~м/с (в вакууме).

Очень длинная строка, явно длиннее 80 символов, со множеством непонятных терминов и прочих слов.

Также вспомним, что
\begin{equation*}
\sin^2 \alpha + \cos^2 \alpha = 1
\end{equation*}

откуда

\begin{equation*}
\sin^2 \alpha = 1 - \cos^2 \alpha
\end{equation*}

\begin{thebibliography}{99}


\Bibitem{1}
\by J.~Smith
\paper Very interesting paper
\jour Any journal
\vol 42
\pages 1--13
\yr 1984


\end{thebibliography}
\end{document}

